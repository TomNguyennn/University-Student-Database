\documentclass[12pt, Times New Romance, margin]{article}
\usepackage{graphicx} % Required for inserting images
\usepackage{amsmath}
\usepackage[a4paper, total={6in, 8in}]{geometry}
\usepackage{float}
\usepackage{listings}
\usepackage{microtype}
\renewcommand{\baselinestretch}{1.5} 
\title{Report}
\author{Nhat Nam Nguyen}
\date{November 2024}

\begin{document}
\maketitle
\section{Structured Data}
\subsection{Ex1}
\begin{lstlisting}
#!/bin/bash
xml_file= $1
csv_file= $2

if $xml_file = "students.xml" then
    echo "student_name,student_id,student_email,programme,year,
    address,contact,module_id,module_name,module_leader,lecturer1,
    lecturer2,faculty,building,room,exam_mark,coursework1,coursework2,
    coursework3" > $csv_file
    grep -oP '(?<=<student>).*?(?=</student>)'
    $xml_file | while read -r student; do
      student_name=$(echo "$student" | grep -oP '
      (?<=<student_name>).*?(?=</student_name>)')
      student_id=$(echo "$student" | grep -oP 
      '(?<=<student_id>).*?(?=</student_id>)')
      student_email=$(echo "$student" | grep -oP 
      '(?<=<student_email>).*?(?=</student_email>)')
      programme=$(echo "$student" | grep -oP 
      '(?<=<programme>).*?(?=</programme>)')
      year=$(echo "$student" | grep -oP 
      '(?<=<year>).*?(?=</year>)')
      address=$(echo "$student" | grep -oP 
      '(?<=<address>).*?(?=</address>)' | sed 's/,//g')
      contact=$(echo "$student" | grep -oP 
      '(?<=<contact>).*?(?=</contact>)')
      module_id=$(echo "$student" | grep -oP
      '(?<=<module_id>).*?(?=</module_id>)')
      module_name=$(echo "$student" | grep -oP 
      '(?<=<module_name>).*?(?=</module_name>)')
      module_leader=$(echo "$student" | grep -oP 
      '(?<=<module_leader>).*?(?=</module_leader>)')
      lecturer1=$(echo "$student" | grep -oP 
      '(?<=<lecturer1>).*?(?=</lecturer1>)')
      lecturer2=$(echo "$student" | grep -oP 
      '(?<=<lecturer2>).*?(?=</lecturer2>)')
      faculty=$(echo "$student" | grep -oP 
      '(?<=<faculty>).*?(?=</faculty>)')
      building=$(echo "$student" | grep -oP 
      '(?<=<building>).*?(?=</building>)')
      room=$(echo "$student" | grep -oP 
      '(?<=<room>).*?(?=</room>)')
      exam_mark=$(echo "$student" | grep -oP 
      '(?<=<exam_mark>).*?(?=</exam_mark>)')
      coursework1=$(echo "$student" | grep -oP 
      '(?<=<coursework1>).*?(?=</coursework1>)')
      coursework2=$(echo "$student" | grep -oP 
      '(?<=<coursework2>).*?(?=</coursework2>)')
      coursework3=$(echo "$student" | grep -oP 
      '(?<=<coursework3>).*?(?=</coursework3>)')
    
      echo "$student_name,$student_id,$student_email,$programme,$year,$address,$contact,$module_id,$module_name,$module_leader,$lecturer1,$lecturer2,$faculty,$building,$room,$exam_mark,$coursework1,$coursework2,$coursework3" >> $csv_file
    done
else
    echo "faculty,building,room,capacity" > $csv_file
    grep -oP '(?<=<faculty>).*?(?=</faculty>)' $xml_file | while read -r faculty; do
      faculty=$(echo "$faculty" | grep -oP '(?<=<faculty>).*?(?=</faculty>)')
      building=$(echo "$faculty" | grep -oP '(?<=<building>).*?(?=</building>)')
      room=$(echo "$faculty" | grep -oP '(?<=<room>).*?(?=</room>)')
      capacity=$(echo "$faculty" | grep -oP '(?<=<capacity>).*?(?=</capacity>)')
      echo "$faculty,$building,$room,$capacity" >> $csv_file
    done

fi

\end{lstlisting}
Using grep to extract the data in the tags $<faculty> and <student> $ 
\subsection{Ex2}
\begin{lstlisting}
    input = $1
    output = $2
    cat input | cut -f1 -d | sort >> $output.txt 


\end{lstlisting}
\section{Relational Model}
\subsection{Ex3}
Faculty.csv(faculty, building, room, capacity)
\newline
Students.csv(student\_name, student\_id, student\_email, programme, year, address, contact, module\_id, module\_name, module\_leader, lecturer1, lecturer2, faculty, building, room, exam\_mark, coursework1, coursework2, coursework3)
\subsection{Ex4}
The minimal set of functional dependencies for Faculty.csv:
\newline
$\{\text{building}, \text{room} \rightarrow \text{capacity}; \text{ building}, \text{room} \rightarrow \text{faculty} \}$
\newline
The minimal sets of functional dependencies for Students.csv:
\newline
$\{\text{building}, \text{room} \rightarrow \text{faculty}; \, \text{student\_name}, \text{student\_id} \rightarrow \text{student\_email}; \, \\
\text{student\_name}, \text{student\_id} \rightarrow \text{programme}; \, \text{student\_name}, \text{student\_id} \rightarrow \text{address}; \, \\
\text{student\_name}, \text{student\_id} \rightarrow \text{contact}; \, \text{module\_id} \rightarrow \text{module\_name}, \text{module\_leader}; \, \\
\text{module\_id}, \text{module\_name} \rightarrow \text{exam\_mark}; \,  \text{student\_name}, \text{student\_id} \rightarrow \text{year}; \, \\
\text{module\_id}, \text{module\_name} \rightarrow \text{lecturer1}; \, \text{module\_id}, \text{module\_name} \rightarrow \text{lecturer2}; \, \}$

\subsection{Ex5}
Possible candidate keys for faculty.csv: \{building, room\}; \{faculty, building, room\} \\
Possible candidate keys for faculty.csv: \{student\_id, module\_id\}; \{student\_email\, module\_id\}; \{student\_id, module\_name\}; \{student\_email, module\_name\}
\subsection{Ex6}
A primary key for faculty.csv: \{building, room\} \\
A primary key for students.csv: \{student\_id\}

\section{Normalization}
\subsection{Ex7}
No, the data are not in the first normal form because there are some repeated attributes in the database, specifically lecturer attribute and coursework attribute. \\
The reason for it to be not in first normal form is Coursework(student\_id, module\_id, exam\_mark, coursework), Lecturer(module\_id, lecturer)
\subsection{Ex8}
By identifying all of the primary keys which is student\_id, module\_id, module\_name decompose, move all of the dependents into a new table
\subsection{Ex9}
All of the partial-key dependencies are: 
\\ $\{(student\_id, module\_id) \rightarrow student\_name, student\_email, programme, year, address, contact \}$
\subsection{Ex10}
Relations students: \\
Fields:
\begin{itemize}
    \item student\_id (Primary Key, INTEGER)
    \item student\_name (TEXT)
    \item student\_email (TEXT)
    \item programme (TEXT)
    \item year (INTEGER)
    \item address (TEXT)
    \item contact (TEXT)
\end{itemize}
Relation modules( \\
Fields:
\begin{itemize}
    \item module\_id (Primary Key, TEXT)
    \item module\_name (TEXT)
    \item module\_leader (TEXT)
    \item faculty (TEXT)
    \item building (TEXT)
    \item room (TEXT)
\end{itemize} ) \\
Relation modules\_lecturers( \\
Fields:
\begin{itemize}
    \item module\_id (Primary Key, TEXT)
    \item module\_lecturers
\end{itemize}
) \\
Relation modules\_exammark(
\begin{itemize}
    \item student\_id (Foreign Key, TEXT)
    \item module\_id (Foreign Key, TEXT)
    \item exam\_mark (INTEGER)  
\end{itemize}
) \\
Relation modules\_coursework(
\begin{itemize}
    \item student\_id (Foreign Key, TEXT)
    \item module\_id (Foreign Key, TEXT)
    \item coursework (INTEGER)  
\end{itemize}
) \\
\subsection{Ex11}
I did use an appropriate primary key while decomposing which are student\_id and module\_id
\subsection{Ex12}
Transitive dependencies: $module\_id \rightarrow building, room \rightarrow faculty$
\subsection{Ex13}
rooms\_buildings(building, room, faculty); modules(module\_id, module\_name, module\_leader, building, room)

\section{Modeling}
\subsection{Ex14}
\begin{table}[H]
        \centering
        \begin{tabular}{|c|c|c|c|c|c|c|}
            \hline
            student\_id & student\_name & student\_email & programme & year & address & contact \\
            \hline
            INTEGER   & TEXT   & TEXT   & TEXT   & INTEGER   & TEXT   & TEXT   \\
            \hline
        \end{tabular}
        \caption{Student}
        \label{tab:sample_table}
    \end{table}
\begin{table}[H]
    \centering
    \begin{tabular}{|c|c|c|c|c|}
        \hline
        module\_id & module\_name & module\_leader & building & room \\
        \hline
        TEXT   & TEXT   & TEXT   & TEXT   & TEXT   \\
        \hline
    \end{tabular}
    \caption{Module}
    \label{tab:sample_table}
\end{table}
\begin{table}[H]
    \centering
    \begin{tabular}{|c|c|}
        \hline
        module\_id & lecturer \\
        \hline
        TEXT   & TEXT  \\
        \hline
    \end{tabular}
    \caption{Lecturer}
    \label{tab:sample_table}
\end{table}
\begin{table}[H]
    \centering
    \begin{tabular}{|c|c|c|}
        \hline
        student\_id & module\_id & coursework \\
        \hline
        INTEGER   & TEXT   & INTEGER \\
        \hline
    \end{tabular}
    \caption{student\_coursework}
    \label{tab:sample_table}
\end{table}
\begin{table}[H]
    \centering
    \begin{tabular}{|c|c|c|}
        \hline
        student\_id & module\_id & exam\_mark \\
        \hline
        INTEGER   & TEXT   & INTEGER \\
        \hline
    \end{tabular}
    \caption{student\_exam\_mark}
    \label{tab:sample_table}
\end{table}
\begin{table}[H]
    \centering
    \begin{tabular}{|c|c|}
        \hline
        Student\_id & Module\_id  \\
        \hline
        INTEGER   & TEXT    \\
        \hline
    \end{tabular}
    \caption{Enrolled}
    \label{tab:Building}
\end{table}
\begin{table}[H]
    \centering
    \begin{tabular}{|c|c|c|}
        \hline
        building & room & faculty \\
        \hline
        TEXT   & TEXT   & TEXT  \\
        \hline
    \end{tabular}
    \caption{Building faculty}
    \label{tab:Building}
\end{table}
\begin{table}[H]
    \centering
    \begin{tabular}{|c|c|c|}
        \hline
        building & room & capacity \\
        \hline
        TEXT   & TEXT   &  INTEGER  \\
        \hline
    \end{tabular}
    \caption{Building capacity}
    \label{tab:Building}
\end{table}

\subsection{Ex15}
\begin{lstlisting}
    .mode csv
    .separator ","
    .import faculty.csv facultycsv
    .import student.csv studentcsv
    .ouput ex15.sql
    .dump
\end{lstlisting}
\subsection{Ex16}
\begin{lstlisting}
    CREATE TABLE student(student_id INTEGER PRIMARY KEY NOT NULL UNIQUE, 
    student_name TEXT NOT NULL, student_email TEXT NOT NULL, 
    programme TEXT NOT NULL, year INTEGER NOT NULL, address TEXT, 
    contact TEXT);
    
    INSERT OR IGNORE INTO student SELECT student_id, student_name, 
    student_email, programme, year, address, contact FROM studentcsv;
    
    CREATE TABLE enrolled(student_id INTEGER PRIMARY KEY NOT NULL, 
    module_id TEXT NOT NULL);
    
    INSERT OR IGNORE INTO student SELECT student_id, 
    module_id FROM studentcsv;
    
    CREATE TABLE module(module_id TEXT PRIMARY KEY NOT NULL UNIQUE, 
    module_name TEXT NOT NULL UNIQUE, module_leader TEXT NOT NULL, 
    building TEXT, room TEXT);
    
    INSERT OR IGNORE INTO module SELECT module_id, 
    module_name, module_leader, building, room FROM studentcsv;
    
    CREATE TABLE lecturer
    (module_id PRIMARY KEY NOT NULL, lecturer TEXT UNIQUE);
    
    INSERT OR IGNORE INTO lecturer 
    SELECT module_id, lecturer1 AS lecturer FROM studentcsv
    UNION ALL 
    SELECT module_id, lecturer2 AS lecturer FROM studentcsv;
    
    CREATE TABLE student_exammark(student_id INTEGER NOT NULL, 
    module_id TEXT NOT NULL, exam_mark INTEGER);
    
    INSERT OR IGNORE INTO student_exammark SELECT student_id, module_id, 
    exam_mark FROM studentcsv;
    
    CREATE TABLE student_coursework(student_id INTEGER NOT NULL, 
    module_id TEXT NOT NULL, coursework INTEGER);
    
    INSERT OR IGNORE INTO student_coursework 
    SELECT student_id, module_id, coursework1 
    AS coursework FROM studentcsv
    UNION ALL 
    SELECT student_id, module_id, coursework2 
    AS coursework FROM studentcsv
    UNION ALL 
    SELECT student_id, module_id, coursework3 
    AS coursework FROM studentcsv;
    
    CREATE TABLE building_room(building TEXT NOT NULL, 
    room TEXT NOT NULL, faculty TEXT);
    INSERT OR IGNORE INTO building_room SELECT building, 
    room, faculty FROM facultycsv;
    
    CREATE TABLE building_capacity(building TEXT NOT NULL, 
    room TEXT NOT NULL, capacity INTEGER);
    
    INSERT OR IGNORE INTO building_capacity SELECT 
    building, room, capacity FROM facultycsv;

\end{lstlisting}
\section{Querying}
\subsection{Ex17}
\begin{lstlisting}
    SELECT building, SUM(capacity) AS total_capacity
    FROM building_capacity
    GROUP BY building;
\end{lstlisting}
\subsection{Ex18}
\begin{lstlisting}
    SELECT student.student_id, student.student_name, 
    AVG(student_exammark.exam_mark) AS average_mark
    FROM student
    JOIN student_exammark ON 
    student.student_id = student_exammark.student_id
    WHERE 
    student.year = 1 AND student.programme = 'Computer Science'
    GROUP BY student.student_id, student.student_name
    ORDER BY average_mark DESC;

\end{lstlisting}
\subsection{Ex19}
\begin{lstlisting}
    SELECT module.module_id, module.module_leader, building_room.faculty, AVG(student_exammark.exam_mark) as average
    FROM module
    JOIN building_room ON module.building = building_room.building 
    AND module.room = building_room.room
    JOIN student_exammark 
    ON module.module_id = student_exammark.module_id
    GROUP BY 
    module.module_id, module.module_leader, building_room.faculty
    HAVING AVG(student_exammark.exam_mark) = (
        SELECT MAX(average_mark)
        FROM (
            SELECT module.module_id, AVG(student_exammark.exam_mark) 
            AS average_mark
            FROM module 
            JOIN student_exammark 
            ON module.module_id = student_exammark.module_id
            WHERE module.building = building_room.building 
            AND module.room = building_room.room
            GROUP BY module.module_id
    )
);


\end{lstlisting}
\subsection{Ex20}
\begin{lstlisting}
    SELECT module.module_id, building_capacity.room, 
    building_capacity.building
    FROM module
    JOIN building_capacity 
    ON module.room = building_capacity.room
    JOIN enrolled 
    ON module.module_id = enrolled.module_id
    GROUP BY 
    module.module_id, building_capacity.building, building_capacity.room 
    HAVING COUNT(enrolled.student_id) > building_capacity.capacity;
\end{lstlisting}
\end{document}
